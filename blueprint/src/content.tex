The proof of the infinitude of primes via the fact that $\log x\le \pi(x)+1$ is the fourth proof of this theorem in Aigner \& Ziegler's "Proofs from THE BOOK", and attributed to Euler. The aim of this formalization is to stay close to the text of the 6th edition of "Proofs from THE BOOK".

The orginal argument by Euler(E72) involves divergent infinite sums and, so, does not satisfy
modern standards of rigor(San2006).

(E72)
Euler, Leonhard, Variae observationes circa series infinitas, Commentarii academiae scientiarum
Petropolitanae 9 (1737), 1744, p. 160-188. Reprinted in Opera Omnia Series I volume 14, p. 216-244.
Also available on line at \verb|www.EulerArchive.org|. 

(San2006)
Sandifer, Ed, How Euler Did It, Infinitely many primes, March 2006, \begin{verbatim}http://eulerarchive.maa.org/hedi/HEDI-2006-03.pdf\end{verbatim}

\begin{definition}
\label{def:nx}
In the following, the variables $n,x$, $n\in\mathbb{N}$, $x\in\mathbb{R}$, always satisfy
$n\le x < n+1$, which is equivalent to $n=\lfloor x\rfloor$. 
\end{definition}

\begin{lemma}
\label{lem:lemma0}
\lean{log_le_harmonic}
\leanok
$$\log x \le \sum_{i=1}^n\frac1i.$$
\end{lemma}
\begin{proof}
\leanok
    We use Mathlib lemma \verb|log_add_one_le_harmonic| which proves the inequality for $x=n+1$. 
\end{proof}

\begin{definition}
\label{def:S}
\lean{S₁}
\leanok
    $$ S_x = \{1\}\ \cup\ \{m\,\big|\, \text{$m$ has only prime factors $p\le x$}\} $$
    $$ = \text{the $\lfloor x\rfloor$- or $n$-smooth numbers} $$
\end{definition}

\begin{lemma}
\label{lem:lemma1}
\lean{H_P4_1}
\uses{def:S}
\leanok
    $$\sum_{k=1}^n\frac1k \le \sum_{m\in S_x}\frac{1}{m}. $$
\end{lemma}
\begin{proof}
All of the inverses of the left sum are contained in the right sum.

In Lean Mathlib, $n$-smooth numbers are defined as numbers all of whose prime factors are less than n.
So, in the Lean proofs, a shift to $n+1$-smooth numbers is necessary.
\end{proof}
The argument in Aigner \& Ziegler, 6th ed. had the set $S_x$ not containing~$1$ which
is problematic, because, while the inverses of all numbers $2\le k \le n$ are also contained
in the right sum, this sum does not have the inverse of $1$, since $1$ does not have prime factors.
So the right
sum has to balance the inequality with the higher-power inverses. The case $n=4$ is a counterexample,
where the l.h.s.~is $\tfrac{25}{12}$ and the r.h.s. is~$2$. Only for $n\ge5$ does the statement seem valid,
although still being messy to prove.


The addition of $\{1\}$ to $S_x$ also leads to a correct statement in the next step.

\begin{lemma}
\label{lem:lemma2}
\lean{H_P4_2}
\uses{def:S}
\leanok
    $$\sum_{m\in S_x}\frac{1}{m} = \underset{p\le x}{\prod_{p\in\mathbb{P}}}\Big(\sum_{k\in\mathbb{N}}\frac{1}{p^k}\Big).$$
\end{lemma}

\begin{lemma}
\label{lem:lemma3}
\lean{H_P4_3}
\leanok
    $$\underset{p\le x}{\prod_{p\in\mathbb{P}}}\Big(\sum_{k\in\mathbb{N}}\frac{1}{p^k}\Big) = \underset{p\le x}{\prod_{p\in\mathbb{P}}}\frac{p}{p-1}$$
\end{lemma}

\begin{lemma}
\label{lem:lemma4-1}
\lean{H_P4_4a}
\leanok
For $k,p \in\mathbb{R}$, $k\ge0$, $p \ge k+1$,
$$\frac{p}{p-1} \le \frac{k+1}{k}.$$
\end{lemma}
\begin{proof}
\leanok
The current proof is by algebraic manipulation of fractions.
\end{proof}

\begin{definition-pre}
\label{def:piReal}
\lean{primeCountingReal}
Given the prime counting function $\pi(n)$, $n\in\mathbb{N}$, we define $\pi(x)$, $x\in\mathbb{R}$ as $\pi(x) = \pi(|\lfloor x\rfloor|)$ for $x\ge2$, and $0$ otherwise.
\end{definition-pre}

\begin{lemma}
\label{lem:lemma4}
\lean{H_P4_4}
\uses{lem:lemma4-1}
$$\underset{p\le x}{\prod_{p\in\mathbb{P}}}\frac{p}{p-1} \le \prod_{k=1}^{\pi(x)}\frac{k+1}{k}.$$
\end{lemma}

\begin{lemma}
\label{lem:lemma5-2-1}
\lean{monotone_primeCountingReal}
\leanok
$\pi(x)$ is monotone.
\end{lemma}
\begin{proof}
\leanok
By cases and the monotonicity of $\pi(n)$ and that of the floor function.
\end{proof}

\begin{lemma}
\label{lem:lemma5-2}
\lean{primeCountingReal_ge_two}
\uses{lem:lemma5-2-1}
\leanok
For $x\ge3$, $\pi(x)\ge2$.
\end{lemma}

\begin{lemma}
\label{lem:lemma5-1}
\lean{prod_Icc_succ_div}
\leanok
For $n\ge2$,
    $$\prod_{k=1}^{n}\frac{k+1}{k} = n+1.$$
\end{lemma}
\begin{proof}
\leanok
We transform into a product over the open interval and use the existing Mathlib4's lemmata in an induction proof.
\end{proof}

\begin{lemma}
\label{lem:lemma5}
\lean{H_P4_5}
\uses{lem:lemma5-1, lem:lemma5-2}
\leanok
For $x\ge3$,
    $$\prod_{k=1}^{\pi(x)}\frac{k+1}{k} = \pi(x)+1.$$
\end{lemma}

\begin{theorem}
\label{thm:log_le_primeCountingReal_add_one}
\lean{log_le_primeCountingReal_add_one}
\uses{lem:lemma0, lem:lemma1, lem:lemma2, lem:lemma3, lem:lemma4, lem:lemma5}
\leanok
For $x\in\mathbb{R}$, $\log x \le \pi(x) +1$.
\end{theorem}

\begin{theorem}
\label{thm:primeCountingReal_unbounded}
\lean{primeCountingReal_unbounded}
\uses{thm:log_le_primeCountingReal_add_one}
\leanok
The prime counting function in the real domain, $\pi:\ \mathbb{R} \mapsto \mathbb{N} \ := \#\big\{p\mid p \in\mathbb{P}, p\le x \big\}$ is unbounded above.
\end{theorem}

\begin{theorem}
\label{thm:infinite_primes}
\lean{infinite_setOf_prime}
\leanok
\uses{thm:primeCountingReal_unbounded}
The set $\mathbb{P} = \big\{p \in \mathbb{N} \ \mid \ p \ \mathrm{prime} \big\}$ is infinite. 
\end{theorem}

